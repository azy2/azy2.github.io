\documentclass[10pt, letterpaper]{article}

\usepackage[margin=0.55in,top=0.25in,bottom=0.1in]{geometry}
\usepackage{hyperref}
\pagenumbering{gobble}
\usepackage{parskip}
\usepackage{tabularx}

\begin{document}
\par

{\centering
\textbf{\LARGE{Benjamin M. Lambeth}}
\smallskip
\hrule \par}

310 E Springfield Ave, Apt. 107, Champaign IL, 61820 \hfill blambet2@illinois.edu \\ \href{http://www.lambethben.com/}{http://www.lambethben.com/} \hfill 217-979-7551

\smallskip
{\centering \subsection*{EDUCATION} \par}

BS, Computer Science \hfill 2015 - 2019 (In Progress) \\
University of Illinois at Urbana Champaign \hfill GPA: 3.82

\smallskip
{\centering \subsection*{INTERNSHIPS} \par}

\textbf{Qualcomm Research and Developement} \hfill May 2017 - Aug 2017 \\
Developed a Regression Test Selection tool that is being used internally
at Qualcomm. The tool speeds up testing by only running the tests that execute
new code and ignoring all other tests. For one project, I brought test execution times down from
approximately 4 hours to minutes or seconds depending on the change. Awarded the
Roberto Padovani Scholarship for this work, which is awarded for the most
substantial work done by an intern that summer. My manager was
Vince Baglin and my mentor was Shashank Khanvilkar.

\textbf{Facebook} \hfill June 2018 - Aug 2018 \\
Worked on HHVM and the Hack language. Created a new datatype, Shapes, in HHVM.
Shapes are designed to operate very similarly to arrays and dicts with the
intention to migrate their functionality into something more typesafe later.
This work spanned 3 months and required substantial understanding of the entire
HHVM JIT Compiler stack. The code I wrote can be found at
\href{https://github.com/facebook/hhvm/commits?author=azy2}{https://github.com/facebook/hhvm/commits?author=azy2}

\smallskip
{\centering \subsection*{RESEARCH} \par}

\textbf{Research Assistant for Darko Marinov and Alex Gyori} \hfill Apr 2016 -
Dec 2016 \\
I did research in the field of Software Testing. I primarily worked on the tool
NonDex  (\href{https://github.com/TestingResearchIllinois/NonDex}{link}), which
is a tool for detecting and debugging wrong assumptions on
under-determined Java APIs. Such assumptions can hurt portability for an
application when they are moved to other environments with a different Java
runtime. NonDex explores different behaviors of under-determined APIs and
reports test failures under different explored behaviors. I also contributed to numerous open-source projects by fixing tests that NonDex flagged.

\smallskip
{\centering \subsection*{PROJECTS} \par}

\textbf{\href{https://github.com/azy2/plan8/tree/master/student-distrib}{Plan8
    (link)}} \par
Plan8 is a small operating system I developed with 2 other students for a class at
The University of Illinois at Urbana-Champaign. The OS supports interupts, an in
memory file system, virtual memory, paging, user space programs, a basic set of
system calls (such as file I/O), task scheduling, a shell, and user signals. I
won 2nd place in the class competition against 50+ other teams.

\textbf{\href{https://github.com/azy2/Overseer/}{OverSeer (link)}} \par
Overseer is a dorm management system I developed with 7 other students over the
course of a semester. Overseer is a full stack web application written in
Python, MySQL, and Bootstrap 4. It allows dorms and residence halls to manage
students, meal plans, packages, and staff in a clean and efficient online system.

\smallskip
{\centering \subsection*{PUBLICATIONS} \par}

A. Gyori, \textbf{B. Lambeth}, A. Shi, O. Legunsen, and D. Marinov. NonDex: A tool for detecting and debugging wrong assumptions on Java API specifications. In \textit{Proceedings of the 24th ACM SIGSOFT International Symposium on the Foundations of Software Engineering (FSE 2016).} Acceptance Rate: 13/32 - 41\%, \href{https://doi.org/10.1145/2950290.2983932}{link}.
\par
A. Gyori, \textbf{B. Lambeth}, S. Khurshid, and D. Marinov. Exploring Underdetermined Specifications using Java PathFinder. In \textit{Java Pathfinder Workshop 2016, Software Engineering Notes (SEN)}, \href{http://mir.cs.illinois.edu/~gyori/pubs/jpf16.pdf}{link}.

\smallskip
{\centering \subsection*{AWARDS/GRANTS} \par}
\begin{tabularx}{\textwidth}{X r}
NSF Travel Grant for attendance at \textit{The International Symposium on the
  Foundations of Software Engineering (FSE) 2016} & \$1,000
\\
  Qualcomm Roberto Padovani Scholarship for most substantial contribution as an intern & \$5,000
\end{tabularx}
\smallskip
\end{document}
